
\documentclass[a4paper]{article}
\usepackage[utf8]{inputenc}
\usepackage[english]{babel}
\usepackage{graphicx}
\usepackage{multicol}
\usepackage{amsmath}
\usepackage{hyperref}
\usepackage{amsthm}
\usepackage{geometry}
\geometry{a4paper} 
\usepackage{fancyhdr}
\usepackage{xcolor}
\begin{document}
\author{\textbf{Fractals}}
\title{\textbf{Problem Set 1 Sloutions}}
\date {\today}
\maketitle
\noindent
\begin{enumerate}
    \item Let $A$, $B$ and $C$ be digits, and assume $A\neq 0$, 
        if $AA + BB + CC = ABC$, what are $A, B$ and $C$?
        \\ \\
        \textbf{Answer:}
        \begin{equation}
            AA + BB + CC = ABC
        \end{equation}
        Note that (1) are digits can be rewritten as 
        $$ 10A + A + 10B + B + 10C + C = 100A+10B+C$$ By Collecting like terms
        $$ B + 10 C = 89 A$$ Since $B \ and \ C$ are digits. $B + 10 C$ is between $0$ and $99$
        which implies $89A$, Hence $A=1$ \\ \\ 
        Note in (1) Adding $A,B,C$ give the digit $C$ in the ones place, which means $A+B$ adds to $10$ \\
        Thus $B=9$. This implies $C=8$

    \item  Find the value of $\dfrac{1}{3^2+1}+
        \dfrac{1}{4^2+1}+\dfrac{1}{5^2+1} + \dots$
        \\ \\
        \textbf{Answer:} \\
        Each term takes the form:
        $$ \dfrac{1}{n^2 + (n-2)} = \dfrac{1}{(n+2)(n-1)}$$
        Using the method of partial fractions, we can rewrite :
        $$ \dfrac{1}{(n+2)(n-1)} = \dfrac{A}{n+2} + \dfrac{B}{(n-1)}$$
        $$ \Rightarrow 1 = A.(n-1) + B.(n+2)$$
        Setting $n=1$ we get $B=\dfrac{1}{3}$ and smiliarly with $n=-2$ we 
        get $A=\dfrac{-1}{3}$. 
        \[
            \dfrac{-1/3}{(n+2)} + \dfrac{1/3}{(n-1)} = \dfrac{1}{3} \times (\dfrac{-1}{(n+2)} + \dfrac{1}{(n-1)})
        \] Hence the sum becomes
        \[
           \displaystyle \frac{1}{3} \times [(\frac{1}{2}-\frac{1}{5}) + (\frac{1}{3}-\frac{1}{6})
           +(\frac{1}{4}-\frac{1}{7}) + (\frac{1}{5} - \frac{1}{8})+  \dots]
        \]
        Thus, it telescopes, and the only terms that do not cancel produce a sum of $\frac{1}{3} \times
        (\frac{1}{2} + \frac{1}{3} + \frac{1}{4}) = \frac{13}{36}$
    \item Let $f(x) = 1 + x + x^2 + x^3 + \dots+ x^{100}$.Find $f^\prime(1)$
        \\ \\
        \textbf{Answer:}
        Note that $f^\prime(x) = 1 + 2x + 3x^2 + \dots + 100x^9
        $, so $f^\prime(1) = 1+2+\dots+100 = \dfrac{100.101}{2} = 5050$
    \item Two reals $x$ and $y$ are such that $x - y = 4$ and $x^3 - y^3 = 28$ compute $xy$ 
        \\ \\
        \textbf{Answer:}
        We have $$28 = x^3 -y^3 = (x-y)(x^2 + xy + y^2) = (x-y)((x-y)^2 + 3xy) = 4 \times(16 + 3xy)$$
        from which $ xy = −3.$
    \item  For each positive integer $n$, let $f(n) = \dfrac{n}{n+1} + \dfrac{n+1}{n}$. Then $f (1) + f (2) + f (3) + 
    \dots + f (10)$ can be expressed as $\dfrac{m}{n}$ where $m$ and $n$ are relatively prime positive integers. Compute $m + n$.
        \\ \\
        \textbf{Answer:}
        \[
            f(n) = \frac{n+1-1}{n+1} + \frac{n+1}{n} = 
        \]
        \[
            (\frac{n+1}{n+1} - \frac{1}{n+1}) + (\frac{n}{n} + \frac{1}{n}) = 2 + \frac{1}{n}
             - \frac{1}{n+1}
        \]
        So 
        \[
            f(1) + f(2) + f(3) + \dots + f(10) = 2.10  + (\frac{1}{1} - \frac{1}{2}) + (\frac{1}{2} - \frac{1}{3}) + \dots
             + (\frac{1}{10} - \frac{1}{11}) = 20 + \frac{10}{11} =  \frac{230}{11}.
        \]
\end{enumerate}
\end{document}
